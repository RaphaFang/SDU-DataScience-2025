% cmd + opt + v
% 不確定報錯因素,但是又能夠透過 short cut 得到 pdf
% ------------------------------------------------------------------------

\documentclass[12pt]{article}

\usepackage[T1]{fontenc}  % 字型編碼:避免非 ASCII 字元錯誤
\usepackage{amsmath, amssymb, amsthm}  % 數學環境
\usepackage{enumitem}                  % 自訂 enumerate 格式

% theorem / proof 環境宣告
\newtheorem{theorem}{Theorem}
\newtheorem{lemma}{Lemma}

\begin{document}

\begin{theorem}
Let $\texttt{num}\in\mathbb{Z}_{>0}$.  
The function \texttt{isPerfectSquare(num)} (binary-search version) returns
\textbf{true} if and only if $\texttt{num}$ is a perfect square.
\end{theorem}

\begin{proof}[Proof by invariant and contradiction]
%
\paragraph{Invariant.}~%
At the start of every loop iteration, if there exists an integer
$r$ such that $r^2=\texttt{num}$, then $r\in[\texttt{low},\,\texttt{high}]$.

\paragraph{Initialization.}~%
Initially $\texttt{low}=1,\,\texttt{high}=\texttt{num}$, so the interval
contains \emph{all} positive integers; the invariant holds vacuously.

\paragraph{Maintenance.}~%
Let $\texttt{mid}=\bigl\lfloor(\texttt{low}+\texttt{high})/2\bigr\rfloor$.

\begin{enumerate}[label=\textbf{Case \arabic*:}, wide=0pt]
  \item $\texttt{mid}^2<\texttt{num}$.  
        Assume for contradiction that the real root $r\le\texttt{mid}$.  
        Then $r^2\le\texttt{mid}^2<\texttt{num}$, contradicting $r^2=\texttt{num}$.  
        Hence $r>\texttt{mid}$ and still lies in
        $[\texttt{mid}+1,\texttt{high}]$ after we set
        $\texttt{low}=\texttt{mid}+1$.
  \item $\texttt{mid}^2>\texttt{num}$.  
        The symmetric argument shows $r<\texttt{mid}$, therefore
        $r\in[\texttt{low},\texttt{mid}-1]$ after we update
        $\texttt{high}=\texttt{mid}-1$.
\end{enumerate}
In each case the invariant is preserved.

\paragraph{Termination.}~%
The loop stops in two possible situations:
\begin{itemize}
  \item $\texttt{mid}^2=\texttt{num}$ --- we return \textbf{true}, and $r=\texttt{mid}$.
  \item $\texttt{low}>\texttt{high}$ --- suppose, for contradiction, that an
        integer root $r$ exists.  Then $r\notin[\texttt{low},\texttt{high}]$,
        violating the invariant.  Hence no such $r$ exists and the function
        correctly returns \textbf{false}.
\end{itemize}
\end{proof}

\end{document}


% \documentclass{article}
% \begin{document}
% Hello, \LaTeX!

% Here is an equation:
% \[
%   E = mc^2
% \]
% \end{document}
